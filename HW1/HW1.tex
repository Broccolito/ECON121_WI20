\PassOptionsToPackage{unicode=true}{hyperref} % options for packages loaded elsewhere
\PassOptionsToPackage{hyphens}{url}
%
\documentclass[
]{article}
\usepackage{lmodern}
\usepackage{amssymb,amsmath}
\usepackage{ifxetex,ifluatex}
\ifnum 0\ifxetex 1\fi\ifluatex 1\fi=0 % if pdftex
  \usepackage[T1]{fontenc}
  \usepackage[utf8]{inputenc}
  \usepackage{textcomp} % provides euro and other symbols
\else % if luatex or xelatex
  \usepackage{unicode-math}
  \defaultfontfeatures{Scale=MatchLowercase}
  \defaultfontfeatures[\rmfamily]{Ligatures=TeX,Scale=1}
\fi
% use upquote if available, for straight quotes in verbatim environments
\IfFileExists{upquote.sty}{\usepackage{upquote}}{}
\IfFileExists{microtype.sty}{% use microtype if available
  \usepackage[]{microtype}
  \UseMicrotypeSet[protrusion]{basicmath} % disable protrusion for tt fonts
}{}
\makeatletter
\@ifundefined{KOMAClassName}{% if non-KOMA class
  \IfFileExists{parskip.sty}{%
    \usepackage{parskip}
  }{% else
    \setlength{\parindent}{0pt}
    \setlength{\parskip}{6pt plus 2pt minus 1pt}}
}{% if KOMA class
  \KOMAoptions{parskip=half}}
\makeatother
\usepackage{xcolor}
\IfFileExists{xurl.sty}{\usepackage{xurl}}{} % add URL line breaks if available
\IfFileExists{bookmark.sty}{\usepackage{bookmark}}{\usepackage{hyperref}}
\hypersetup{
  pdftitle={ECON 121 HW1},
  pdfauthor={Wanjun Gu},
  pdfborder={0 0 0},
  breaklinks=true}
\urlstyle{same}  % don't use monospace font for urls
\usepackage[margin=1in]{geometry}
\usepackage{color}
\usepackage{fancyvrb}
\newcommand{\VerbBar}{|}
\newcommand{\VERB}{\Verb[commandchars=\\\{\}]}
\DefineVerbatimEnvironment{Highlighting}{Verbatim}{commandchars=\\\{\}}
% Add ',fontsize=\small' for more characters per line
\usepackage{framed}
\definecolor{shadecolor}{RGB}{248,248,248}
\newenvironment{Shaded}{\begin{snugshade}}{\end{snugshade}}
\newcommand{\AlertTok}[1]{\textcolor[rgb]{0.94,0.16,0.16}{#1}}
\newcommand{\AnnotationTok}[1]{\textcolor[rgb]{0.56,0.35,0.01}{\textbf{\textit{#1}}}}
\newcommand{\AttributeTok}[1]{\textcolor[rgb]{0.77,0.63,0.00}{#1}}
\newcommand{\BaseNTok}[1]{\textcolor[rgb]{0.00,0.00,0.81}{#1}}
\newcommand{\BuiltInTok}[1]{#1}
\newcommand{\CharTok}[1]{\textcolor[rgb]{0.31,0.60,0.02}{#1}}
\newcommand{\CommentTok}[1]{\textcolor[rgb]{0.56,0.35,0.01}{\textit{#1}}}
\newcommand{\CommentVarTok}[1]{\textcolor[rgb]{0.56,0.35,0.01}{\textbf{\textit{#1}}}}
\newcommand{\ConstantTok}[1]{\textcolor[rgb]{0.00,0.00,0.00}{#1}}
\newcommand{\ControlFlowTok}[1]{\textcolor[rgb]{0.13,0.29,0.53}{\textbf{#1}}}
\newcommand{\DataTypeTok}[1]{\textcolor[rgb]{0.13,0.29,0.53}{#1}}
\newcommand{\DecValTok}[1]{\textcolor[rgb]{0.00,0.00,0.81}{#1}}
\newcommand{\DocumentationTok}[1]{\textcolor[rgb]{0.56,0.35,0.01}{\textbf{\textit{#1}}}}
\newcommand{\ErrorTok}[1]{\textcolor[rgb]{0.64,0.00,0.00}{\textbf{#1}}}
\newcommand{\ExtensionTok}[1]{#1}
\newcommand{\FloatTok}[1]{\textcolor[rgb]{0.00,0.00,0.81}{#1}}
\newcommand{\FunctionTok}[1]{\textcolor[rgb]{0.00,0.00,0.00}{#1}}
\newcommand{\ImportTok}[1]{#1}
\newcommand{\InformationTok}[1]{\textcolor[rgb]{0.56,0.35,0.01}{\textbf{\textit{#1}}}}
\newcommand{\KeywordTok}[1]{\textcolor[rgb]{0.13,0.29,0.53}{\textbf{#1}}}
\newcommand{\NormalTok}[1]{#1}
\newcommand{\OperatorTok}[1]{\textcolor[rgb]{0.81,0.36,0.00}{\textbf{#1}}}
\newcommand{\OtherTok}[1]{\textcolor[rgb]{0.56,0.35,0.01}{#1}}
\newcommand{\PreprocessorTok}[1]{\textcolor[rgb]{0.56,0.35,0.01}{\textit{#1}}}
\newcommand{\RegionMarkerTok}[1]{#1}
\newcommand{\SpecialCharTok}[1]{\textcolor[rgb]{0.00,0.00,0.00}{#1}}
\newcommand{\SpecialStringTok}[1]{\textcolor[rgb]{0.31,0.60,0.02}{#1}}
\newcommand{\StringTok}[1]{\textcolor[rgb]{0.31,0.60,0.02}{#1}}
\newcommand{\VariableTok}[1]{\textcolor[rgb]{0.00,0.00,0.00}{#1}}
\newcommand{\VerbatimStringTok}[1]{\textcolor[rgb]{0.31,0.60,0.02}{#1}}
\newcommand{\WarningTok}[1]{\textcolor[rgb]{0.56,0.35,0.01}{\textbf{\textit{#1}}}}
\usepackage{longtable,booktabs}
% Allow footnotes in longtable head/foot
\IfFileExists{footnotehyper.sty}{\usepackage{footnotehyper}}{\usepackage{footnote}}
\makesavenoteenv{longtable}
\usepackage{graphicx,grffile}
\makeatletter
\def\maxwidth{\ifdim\Gin@nat@width>\linewidth\linewidth\else\Gin@nat@width\fi}
\def\maxheight{\ifdim\Gin@nat@height>\textheight\textheight\else\Gin@nat@height\fi}
\makeatother
% Scale images if necessary, so that they will not overflow the page
% margins by default, and it is still possible to overwrite the defaults
% using explicit options in \includegraphics[width, height, ...]{}
\setkeys{Gin}{width=\maxwidth,height=\maxheight,keepaspectratio}
\setlength{\emergencystretch}{3em}  % prevent overfull lines
\providecommand{\tightlist}{%
  \setlength{\itemsep}{0pt}\setlength{\parskip}{0pt}}
\setcounter{secnumdepth}{-2}
% Redefines (sub)paragraphs to behave more like sections
\ifx\paragraph\undefined\else
  \let\oldparagraph\paragraph
  \renewcommand{\paragraph}[1]{\oldparagraph{#1}\mbox{}}
\fi
\ifx\subparagraph\undefined\else
  \let\oldsubparagraph\subparagraph
  \renewcommand{\subparagraph}[1]{\oldsubparagraph{#1}\mbox{}}
\fi

% set default figure placement to htbp
\makeatletter
\def\fps@figure{htbp}
\makeatother


\title{ECON 121 HW1}
\author{Wanjun Gu}
\date{1/31/2020}

\begin{document}
\maketitle

\hypertarget{import-all-the-libraries-and-data}{%
\subsubsection{Import all the libraries and
data}\label{import-all-the-libraries-and-data}}

\begin{Shaded}
\begin{Highlighting}[]
\NormalTok{knitr}\OperatorTok{::}\NormalTok{opts_chunk}\OperatorTok{$}\KeywordTok{set}\NormalTok{(}\DataTypeTok{echo =} \OtherTok{TRUE}\NormalTok{)}
\KeywordTok{library}\NormalTok{(foreign)}
\KeywordTok{library}\NormalTok{(readxl)}
\KeywordTok{library}\NormalTok{(readstata13)}
\KeywordTok{library}\NormalTok{(knitr)}
\KeywordTok{library}\NormalTok{(ggplot2)}
\NormalTok{ps1_cps =}\StringTok{ }\KeywordTok{read_xlsx}\NormalTok{(}\DataTypeTok{path =} \StringTok{"ps1_cps.xlsx"}\NormalTok{)}
\NormalTok{nlsy =}\StringTok{ }\KeywordTok{read.dta13}\NormalTok{(}\DataTypeTok{file =} \StringTok{"nlsy79.dta"}\NormalTok{)}
\end{Highlighting}
\end{Shaded}

\hypertarget{question-1}{%
\subsection{Question 1}\label{question-1}}

In the Mincerian Wage Equation:
\(ln(w_{i}) = \beta_{0} + \beta_{1}ed_{i} + \beta_{3}exper_{i} + \beta_{4}exper^{2} + \epsilon_{i}\),
\(\beta_1\) means the education wage return given the same years of
experience. Or, with same years of experience, how much more (in log
scale) does the individuals with one more year of education earn. The
reason why experience has a square term \(exper^2\) is because people
expect the return of experience to be non-linear. That is, people would
normally expect the more experience one tends to have, the more valuable
the experience is.

\hypertarget{question-2}{%
\subsection{Question 2}\label{question-2}}

\begin{Shaded}
\begin{Highlighting}[]
\CommentTok{# Data Processing}
\NormalTok{cps =}\StringTok{ }\KeywordTok{as.data.frame}\NormalTok{(}\KeywordTok{read_excel}\NormalTok{(}\DataTypeTok{path =} \StringTok{"ps1_cps.xlsx"}\NormalTok{))}
\NormalTok{educ_yr =}\StringTok{ }\KeywordTok{cbind}\NormalTok{(}\KeywordTok{c}\NormalTok{(}\StringTok{"Associate's degree, occupational/vocational program"}\NormalTok{,}
                  \StringTok{"Master's degree"}\NormalTok{                                    ,}
                  \StringTok{"Grades 7 or 8"}\NormalTok{                                      ,}
                  \StringTok{"High school diploma or equivalent"}\NormalTok{                  ,}
                  \StringTok{"Bachelor's degree"}\NormalTok{                                  ,}
                  \StringTok{"Some college but no degree"}\NormalTok{                         ,}
                  \StringTok{"Doctorate degree"}\NormalTok{                                   ,}
                  \StringTok{"12th grade, no diploma"}\NormalTok{                             ,}
                  \StringTok{"Associate's degree, academic program"}\NormalTok{               ,}
                  \StringTok{"Grade 10"}\NormalTok{                                           ,}
                  \StringTok{"Grade 11"}\NormalTok{                                           ,}
                  \StringTok{"None or preschool"}\NormalTok{                                  ,}
                  \StringTok{"Professional school degree"}\NormalTok{                         ,}
                  \StringTok{"Grade 9"}\NormalTok{                                            ,}
                  \StringTok{"Grades 1, 2, 3, or 4"}\NormalTok{                               ,}
                  \StringTok{"Grades 5 or 6"}\NormalTok{  ), }
                \KeywordTok{c}\NormalTok{(}\DecValTok{16}\NormalTok{, }\DecValTok{18}\NormalTok{, }\FloatTok{7.5}\NormalTok{, }\DecValTok{12}\NormalTok{, }\DecValTok{16}\NormalTok{, }\DecValTok{15}\NormalTok{,}
                  \DecValTok{22}\NormalTok{, }\DecValTok{12}\NormalTok{, }\DecValTok{14}\NormalTok{, }\DecValTok{10}\NormalTok{, }\DecValTok{11}\NormalTok{, }\DecValTok{0}\NormalTok{,}
                  \DecValTok{14}\NormalTok{, }\DecValTok{9}\NormalTok{, }\FloatTok{2.5}\NormalTok{, }\FloatTok{5.5}\NormalTok{))}

\NormalTok{cps}\OperatorTok{$}\NormalTok{uhrsworkt =}\StringTok{ }\KeywordTok{as.numeric}\NormalTok{(cps}\OperatorTok{$}\NormalTok{uhrsworkt) }\CommentTok{# Assign NA if work hour varies}
\end{Highlighting}
\end{Shaded}

\begin{verbatim}
## Warning: NAs introduced by coercion
\end{verbatim}

\begin{Shaded}
\begin{Highlighting}[]
\NormalTok{assign_educ =}\StringTok{ }\ControlFlowTok{function}\NormalTok{(x)\{}
  \KeywordTok{return}\NormalTok{(educ_yr[}\KeywordTok{which}\NormalTok{(x }\OperatorTok{==}\StringTok{ }\NormalTok{educ_yr[,}\DecValTok{1}\NormalTok{]),}\DecValTok{2}\NormalTok{])}
\NormalTok{\}}
\NormalTok{cps}\OperatorTok{$}\NormalTok{educ =}\StringTok{ }\KeywordTok{as.numeric}\NormalTok{(}\KeywordTok{sapply}\NormalTok{(cps}\OperatorTok{$}\NormalTok{educ, assign_educ))}
\NormalTok{cps}\OperatorTok{$}\NormalTok{age =}\StringTok{ }\KeywordTok{as.numeric}\NormalTok{(cps}\OperatorTok{$}\NormalTok{age)}
\NormalTok{cps}\OperatorTok{$}\NormalTok{sex =}\StringTok{ }\KeywordTok{as.factor}\NormalTok{(cps}\OperatorTok{$}\NormalTok{sex)}
\NormalTok{cps}\OperatorTok{$}\NormalTok{exper =}\StringTok{ }\NormalTok{cps}\OperatorTok{$}\NormalTok{age }\OperatorTok{-}\StringTok{ }\NormalTok{cps}\OperatorTok{$}\NormalTok{educ }\OperatorTok{-}\StringTok{ }\DecValTok{5}
\NormalTok{cps}\OperatorTok{$}\NormalTok{exper2 =}\StringTok{ }\NormalTok{cps}\OperatorTok{$}\NormalTok{exper}\OperatorTok{^}\DecValTok{2}
\NormalTok{cps}\OperatorTok{$}\NormalTok{white =}\StringTok{ }\KeywordTok{as.numeric}\NormalTok{(cps}\OperatorTok{$}\NormalTok{race }\OperatorTok{==}\StringTok{ "White"}\NormalTok{)}
\NormalTok{cps}\OperatorTok{$}\NormalTok{black =}\StringTok{ }\KeywordTok{as.numeric}\NormalTok{(cps}\OperatorTok{$}\NormalTok{race }\OperatorTok{==}\StringTok{ "Black/Negro"}\NormalTok{)}
\NormalTok{cps}\OperatorTok{$}\NormalTok{other =}\StringTok{ }\KeywordTok{as.numeric}\NormalTok{(cps}\OperatorTok{$}\NormalTok{race }\OperatorTok{!=}\StringTok{ "White"} \OperatorTok{&}\StringTok{ }\NormalTok{cps}\OperatorTok{$}\NormalTok{race }\OperatorTok{!=}\StringTok{ "Black/Negro"}\NormalTok{)}
\NormalTok{cps}\OperatorTok{$}\NormalTok{race =}\StringTok{ }\KeywordTok{as.factor}\NormalTok{(cps}\OperatorTok{$}\NormalTok{race)}
\NormalTok{cps}\OperatorTok{$}\NormalTok{hwage =}\StringTok{ }\KeywordTok{log}\NormalTok{(cps}\OperatorTok{$}\NormalTok{incwage }\OperatorTok{/}\StringTok{ }\NormalTok{(cps}\OperatorTok{$}\NormalTok{uhrsworkt }\OperatorTok{*}\StringTok{ }\NormalTok{cps}\OperatorTok{$}\NormalTok{wkswork1))}
\NormalTok{cps}\OperatorTok{$}\NormalTok{hwage[cps}\OperatorTok{$}\NormalTok{hwage }\OperatorTok{==}\StringTok{ }\OperatorTok{-}\OtherTok{Inf}\NormalTok{] =}\StringTok{ }\DecValTok{0}
\NormalTok{cps =}\StringTok{ }\KeywordTok{na.omit}\NormalTok{(cps) }\CommentTok{# Get rid of all the NA values}
\NormalTok{cps =}\StringTok{ }\NormalTok{cps[cps}\OperatorTok{$}\NormalTok{uhrsworkt }\OperatorTok{>=}\StringTok{ }\DecValTok{35} \OperatorTok{&}\StringTok{ }\NormalTok{cps}\OperatorTok{$}\NormalTok{wkswork1 }\OperatorTok{>=}\StringTok{ }\DecValTok{50}\NormalTok{,] }\CommentTok{# Get rid of part-time}
\KeywordTok{kable}\NormalTok{(}\KeywordTok{head}\NormalTok{(cps)) }\CommentTok{# Display data}
\end{Highlighting}
\end{Shaded}

\begin{longtable}[]{@{}lrllrrrrrrrrrr@{}}
\toprule
& age & sex & race & uhrsworkt & educ & wkswork1 & incwage & exper &
exper2 & white & black & other & hwage\tabularnewline
\midrule
\endhead
2 & 55 & Female & White & 40 & 18 & 52 & 56000 & 32 & 1024 & 1 & 0 & 0 &
3.292984\tabularnewline
6 & 59 & Male & White & 50 & 12 & 52 & 50000 & 42 & 1764 & 1 & 0 & 0 &
2.956512\tabularnewline
7 & 29 & Male & White & 45 & 12 & 52 & 40000 & 12 & 144 & 1 & 0 & 0 &
2.838729\tabularnewline
8 & 30 & Female & White & 40 & 16 & 52 & 30000 & 9 & 81 & 1 & 0 & 0 &
2.668830\tabularnewline
9 & 40 & Male & White & 50 & 15 & 52 & 0 & 20 & 400 & 1 & 0 & 0 &
0.000000\tabularnewline
10 & 46 & Female & White & 40 & 16 & 52 & 45000 & 25 & 625 & 1 & 0 & 0 &
3.074295\tabularnewline
\bottomrule
\end{longtable}

\begin{Shaded}
\begin{Highlighting}[]
\KeywordTok{summary}\NormalTok{(cps) }\CommentTok{# Summarize data}
\end{Highlighting}
\end{Shaded}

\begin{verbatim}
##       age            sex                                    race      
##  Min.   :25.00   Female:22495   White                         :39825  
##  1st Qu.:35.00   Male  :28784   Black/Negro                   : 5923  
##  Median :43.00                  Asian only                    : 3717  
##  Mean   :43.41                  American Indian/Aleut/Eskimo  :  634  
##  3rd Qu.:52.00                  White-American Indian         :  342  
##  Max.   :64.00                  Hawaiian/Pacific Islander only:  268  
##                                 (Other)                       :  570  
##    uhrsworkt          educ          wkswork1        incwage       
##  Min.   : 35.0   Min.   : 0.00   Min.   :50.00   Min.   :      0  
##  1st Qu.: 40.0   1st Qu.:12.00   1st Qu.:52.00   1st Qu.:  30000  
##  Median : 40.0   Median :15.00   Median :52.00   Median :  48500  
##  Mean   : 43.5   Mean   :14.57   Mean   :51.95   Mean   :  64346  
##  3rd Qu.: 45.0   3rd Qu.:16.00   3rd Qu.:52.00   3rd Qu.:  75002  
##  Max.   :170.0   Max.   :22.00   Max.   :52.00   Max.   :1609999  
##                                                                   
##      exper           exper2           white            black       
##  Min.   :-2.00   Min.   :   0.0   Min.   :0.0000   Min.   :0.0000  
##  1st Qu.:15.00   1st Qu.: 225.0   1st Qu.:1.0000   1st Qu.:0.0000  
##  Median :23.00   Median : 529.0   Median :1.0000   Median :0.0000  
##  Mean   :23.84   Mean   : 691.6   Mean   :0.7766   Mean   :0.1155  
##  3rd Qu.:33.00   3rd Qu.:1089.0   3rd Qu.:1.0000   3rd Qu.:0.0000  
##  Max.   :58.00   Max.   :3364.0   Max.   :1.0000   Max.   :1.0000  
##                                                                    
##      other            hwage       
##  Min.   :0.0000   Min.   :-6.254  
##  1st Qu.:0.0000   1st Qu.: 2.668  
##  Median :0.0000   Median : 3.074  
##  Mean   :0.1079   Mean   : 2.996  
##  3rd Qu.:0.0000   3rd Qu.: 3.520  
##  Max.   :1.0000   Max.   : 6.428  
## 
\end{verbatim}

\hypertarget{question-3}{%
\subsection{Question 3}\label{question-3}}

\begin{Shaded}
\begin{Highlighting}[]
\KeywordTok{kable}\NormalTok{(}\KeywordTok{summary}\NormalTok{(}\KeywordTok{lm}\NormalTok{(}\DataTypeTok{data =}\NormalTok{ cps, hwage }\OperatorTok{~}\StringTok{ }\NormalTok{educ }\OperatorTok{+}\StringTok{ }\NormalTok{exper }\OperatorTok{+}\StringTok{ }\NormalTok{exper2))}\OperatorTok{$}\NormalTok{coefficient)}
\end{Highlighting}
\end{Shaded}

\begin{longtable}[]{@{}lrrrr@{}}
\toprule
& Estimate & Std. Error & t value &
Pr(\textgreater{}\textbar{}t\textbar{})\tabularnewline
\midrule
\endhead
(Intercept) & 1.2566103 & 0.0292396 & 42.976331 & 0\tabularnewline
educ & 0.1028490 & 0.0014890 & 69.072179 & 0\tabularnewline
exper & 0.0181678 & 0.0016023 & 11.338398 & 0\tabularnewline
exper2 & -0.0002787 & 0.0000319 & -8.741979 & 0\tabularnewline
\bottomrule
\end{longtable}

Based on the regression results, the return of education is \(0.1\%\)
increase of wage for one year of education.

\hypertarget{question-4}{%
\subsection{Question 4}\label{question-4}}

\begin{Shaded}
\begin{Highlighting}[]
\KeywordTok{kable}\NormalTok{(}\KeywordTok{summary}\NormalTok{(}\KeywordTok{lm}\NormalTok{(}\DataTypeTok{data =}\NormalTok{ cps, hwage }\OperatorTok{~}\StringTok{ }\NormalTok{white }\OperatorTok{+}\StringTok{ }\NormalTok{black }\OperatorTok{+}\StringTok{ }\NormalTok{sex }\OperatorTok{+}\StringTok{ }\NormalTok{educ }\OperatorTok{+}\StringTok{ }\NormalTok{exper }\OperatorTok{+}\StringTok{ }\NormalTok{exper2))}\OperatorTok{$}\NormalTok{coefficient)}
\end{Highlighting}
\end{Shaded}

\begin{longtable}[]{@{}lrrrr@{}}
\toprule
& Estimate & Std. Error & t value &
Pr(\textgreater{}\textbar{}t\textbar{})\tabularnewline
\midrule
\endhead
(Intercept) & 1.1530248 & 0.0320641 & 35.959958 &
0.0000000\tabularnewline
white & -0.0214552 & 0.0129493 & -1.656865 & 0.0975529\tabularnewline
black & -0.1340282 & 0.0168742 & -7.942800 & 0.0000000\tabularnewline
sexMale & 0.1875892 & 0.0080700 & 23.245342 & 0.0000000\tabularnewline
educ & 0.1057297 & 0.0014885 & 71.032201 & 0.0000000\tabularnewline
exper & 0.0169517 & 0.0015935 & 10.637776 & 0.0000000\tabularnewline
exper2 & -0.0002535 & 0.0000317 & -7.995401 & 0.0000000\tabularnewline
\bottomrule
\end{longtable}

The regression result shows that after controling for sex and race, the
coefficient on education becomes more significant. Also, the sex and age
variable themselves are significantly correlated with wage. This
indicates that sex and age differences also explains variations in wage
return of education and they also explains wage differences.

\hypertarget{question-5}{%
\subsection{Question 5}\label{question-5}}

\begin{Shaded}
\begin{Highlighting}[]
\NormalTok{l =}\StringTok{ }\KeywordTok{summary}\NormalTok{(}\KeywordTok{lm}\NormalTok{(}\DataTypeTok{data =}\NormalTok{ cps, hwage }\OperatorTok{~}\StringTok{ }\NormalTok{white }\OperatorTok{+}\StringTok{ }\NormalTok{black }\OperatorTok{+}\StringTok{ }\NormalTok{sex }\OperatorTok{+}\StringTok{ }\NormalTok{educ }\OperatorTok{+}\StringTok{ }\NormalTok{exper }\OperatorTok{+}\StringTok{ }\NormalTok{exper2))}\OperatorTok{$}\NormalTok{coefficient}
\NormalTok{white_coef =}\StringTok{ }\NormalTok{l[}\DecValTok{2}\NormalTok{,}\DecValTok{1}\NormalTok{]}
\NormalTok{white_se =}\StringTok{ }\NormalTok{l[}\DecValTok{2}\NormalTok{,}\DecValTok{2}\NormalTok{]}
\NormalTok{male_coef =}\StringTok{ }\NormalTok{l[}\DecValTok{4}\NormalTok{,}\DecValTok{1}\NormalTok{]}
\NormalTok{male_sd =}\StringTok{ }\NormalTok{l[}\DecValTok{4}\NormalTok{,}\DecValTok{2}\NormalTok{]}
\NormalTok{se =}\StringTok{ }\NormalTok{(white_se}\OperatorTok{^}\DecValTok{2}\OperatorTok{+}\NormalTok{male_sd}\OperatorTok{^}\DecValTok{2}\NormalTok{)}\OperatorTok{^}\FloatTok{0.5}
\NormalTok{t_score =}\StringTok{ }\NormalTok{(white_coef }\OperatorTok{-}\StringTok{ }\NormalTok{male_coef)}\OperatorTok{/}\NormalTok{se}
\KeywordTok{print}\NormalTok{(t_score)}
\end{Highlighting}
\end{Shaded}

\begin{verbatim}
## [1] -13.70059
\end{verbatim}

As the result shows, t statistics is way greater than 1.96, therefore
the differnece is significant.

\hypertarget{question-6}{%
\subsection{Question 6}\label{question-6}}

\begin{Shaded}
\begin{Highlighting}[]
\NormalTok{cps_male =}\StringTok{ }\NormalTok{cps[cps}\OperatorTok{$}\NormalTok{sex }\OperatorTok{==}\StringTok{ "Male"}\NormalTok{,]}
\NormalTok{cps_female =}\StringTok{ }\NormalTok{cps[cps}\OperatorTok{$}\NormalTok{sex }\OperatorTok{==}\StringTok{ "Female"}\NormalTok{,]}
\NormalTok{ml =}\StringTok{ }\KeywordTok{summary}\NormalTok{(}\KeywordTok{lm}\NormalTok{(}\DataTypeTok{data =}\NormalTok{ cps_male, hwage }\OperatorTok{~}\StringTok{ }\NormalTok{white }\OperatorTok{+}\StringTok{ }\NormalTok{black }\OperatorTok{+}\StringTok{ }\NormalTok{educ }\OperatorTok{+}\StringTok{ }\NormalTok{exper }\OperatorTok{+}\StringTok{ }\NormalTok{exper2))}\OperatorTok{$}\NormalTok{coefficient}
\NormalTok{fl =}\StringTok{ }\KeywordTok{summary}\NormalTok{(}\KeywordTok{lm}\NormalTok{(}\DataTypeTok{data =}\NormalTok{ cps_female, hwage }\OperatorTok{~}\StringTok{ }\NormalTok{white }\OperatorTok{+}\StringTok{ }\NormalTok{black }\OperatorTok{+}\StringTok{ }\NormalTok{educ }\OperatorTok{+}\StringTok{ }\NormalTok{exper }\OperatorTok{+}\StringTok{ }\NormalTok{exper2))}\OperatorTok{$}\NormalTok{coefficient}
\KeywordTok{print}\NormalTok{(}\StringTok{"Regression results for males"}\NormalTok{)}
\end{Highlighting}
\end{Shaded}

\begin{verbatim}
## [1] "Regression results for males"
\end{verbatim}

\begin{Shaded}
\begin{Highlighting}[]
\KeywordTok{kable}\NormalTok{(ml)}
\end{Highlighting}
\end{Shaded}

\begin{longtable}[]{@{}lrrrr@{}}
\toprule
& Estimate & Std. Error & t value &
Pr(\textgreater{}\textbar{}t\textbar{})\tabularnewline
\midrule
\endhead
(Intercept) & 1.3543112 & 0.0446224 & 30.350502 &
0.0000000\tabularnewline
white & -0.0269620 & 0.0188256 & -1.432198 & 0.1520979\tabularnewline
black & -0.1629416 & 0.0255723 & -6.371796 & 0.0000000\tabularnewline
educ & 0.1026864 & 0.0020453 & 50.206012 & 0.0000000\tabularnewline
exper & 0.0208734 & 0.0023426 & 8.910550 & 0.0000000\tabularnewline
exper2 & -0.0003353 & 0.0000461 & -7.279945 & 0.0000000\tabularnewline
\bottomrule
\end{longtable}

\begin{Shaded}
\begin{Highlighting}[]
\KeywordTok{print}\NormalTok{(}\StringTok{"Regression results for females"}\NormalTok{)}
\end{Highlighting}
\end{Shaded}

\begin{verbatim}
## [1] "Regression results for females"
\end{verbatim}

\begin{Shaded}
\begin{Highlighting}[]
\KeywordTok{kable}\NormalTok{(fl)}
\end{Highlighting}
\end{Shaded}

\begin{longtable}[]{@{}lrrrr@{}}
\toprule
& Estimate & Std. Error & t value &
Pr(\textgreater{}\textbar{}t\textbar{})\tabularnewline
\midrule
\endhead
(Intercept) & 1.1138004 & 0.0436557 & 25.5132882 &
0.0000000\tabularnewline
white & -0.0167805 & 0.0171791 & -0.9767937 & 0.3286818\tabularnewline
black & -0.1068967 & 0.0214618 & -4.9807839 & 0.0000006\tabularnewline
educ & 0.1106325 & 0.0021375 & 51.7574453 & 0.0000000\tabularnewline
exper & 0.0122206 & 0.0020913 & 5.8434101 & 0.0000000\tabularnewline
exper2 & -0.0001500 & 0.0000422 & -3.5521645 & 0.0003829\tabularnewline
\bottomrule
\end{longtable}

Although there is no way to synthesize the standard error of the two
coefficients of male and female since the samples are innately
different, we can use bootstrap to determine the standard deviation of
the difference between the two samples.

\begin{Shaded}
\begin{Highlighting}[]
\CommentTok{# Construct a bootstrap engine}
\NormalTok{dl_list =}\StringTok{ }\KeywordTok{vector}\NormalTok{()}
\ControlFlowTok{for}\NormalTok{(i }\ControlFlowTok{in} \DecValTok{1}\OperatorTok{:}\DecValTok{1000}\NormalTok{)\{}
\NormalTok{  cps_male_sample =}\StringTok{ }\NormalTok{cps_male[}\KeywordTok{sample}\NormalTok{(}\DecValTok{1}\OperatorTok{:}\KeywordTok{dim}\NormalTok{(cps_male)[}\DecValTok{1}\NormalTok{], }\DecValTok{1000}\NormalTok{, }\DataTypeTok{replace =} \OtherTok{TRUE}\NormalTok{),]}
\NormalTok{  cps_female_sample =}\StringTok{ }\NormalTok{cps_male[}\KeywordTok{sample}\NormalTok{(}\DecValTok{1}\OperatorTok{:}\KeywordTok{dim}\NormalTok{(cps_female)[}\DecValTok{1}\NormalTok{], }\DecValTok{1000}\NormalTok{, }\DataTypeTok{replace =} \OtherTok{TRUE}\NormalTok{),]}
\NormalTok{  ml =}\StringTok{ }\KeywordTok{summary}\NormalTok{(}\KeywordTok{lm}\NormalTok{(}\DataTypeTok{data =}\NormalTok{ cps_male_sample, hwage }\OperatorTok{~}\StringTok{ }\NormalTok{white }\OperatorTok{+}\StringTok{ }\NormalTok{black }\OperatorTok{+}\StringTok{ }\NormalTok{educ }\OperatorTok{+}\StringTok{ }\NormalTok{exper }\OperatorTok{+}\StringTok{ }\NormalTok{exper2))}\OperatorTok{$}\NormalTok{coefficient}
\NormalTok{  fl =}\StringTok{ }\KeywordTok{summary}\NormalTok{(}\KeywordTok{lm}\NormalTok{(}\DataTypeTok{data =}\NormalTok{ cps_female_sample, hwage }\OperatorTok{~}\StringTok{ }\NormalTok{white }\OperatorTok{+}\StringTok{ }\NormalTok{black }\OperatorTok{+}\StringTok{ }\NormalTok{educ }\OperatorTok{+}\StringTok{ }\NormalTok{exper }\OperatorTok{+}\StringTok{ }\NormalTok{exper2))}\OperatorTok{$}\NormalTok{coefficient}
\NormalTok{  dl =}\StringTok{ }\NormalTok{ml }\OperatorTok{-}\StringTok{ }\NormalTok{fl}
\NormalTok{  dl_list =}\StringTok{ }\KeywordTok{c}\NormalTok{(dl_list, dl)}
\NormalTok{\}}
\KeywordTok{hist}\NormalTok{(dl, }\DataTypeTok{breaks =} \DecValTok{50}\NormalTok{, }\DataTypeTok{main =} \StringTok{"Distribution of the coefficient difference"}\NormalTok{,}
     \DataTypeTok{xlab =} \StringTok{"Coefficient difference"}\NormalTok{)}
\end{Highlighting}
\end{Shaded}

\includegraphics{HW1_files/figure-latex/Q6b-1.pdf}

\begin{Shaded}
\begin{Highlighting}[]
\KeywordTok{print}\NormalTok{(}\KeywordTok{paste0}\NormalTok{(}\StringTok{"Mean Difference: "}\NormalTok{, }\KeywordTok{mean}\NormalTok{(dl)))}
\end{Highlighting}
\end{Shaded}

\begin{verbatim}
## [1] "Mean Difference: -0.116987640390738"
\end{verbatim}

\begin{Shaded}
\begin{Highlighting}[]
\KeywordTok{print}\NormalTok{(}\KeywordTok{paste0}\NormalTok{(}\StringTok{"Difference SE: "}\NormalTok{, }\KeywordTok{sd}\NormalTok{(dl)))}
\end{Highlighting}
\end{Shaded}

\begin{verbatim}
## [1] "Difference SE: 0.74265257887611"
\end{verbatim}

As shown in the bootstap result, the difference is not sigificant.
Therefore, the education returns in male and female seperately are not
statistically significant.

\hypertarget{question-7}{%
\subsection{Question 7}\label{question-7}}

\begin{Shaded}
\begin{Highlighting}[]
\NormalTok{cps}\OperatorTok{$}\NormalTok{male =}\StringTok{ }\KeywordTok{as.numeric}\NormalTok{(}\KeywordTok{as.factor}\NormalTok{(cps}\OperatorTok{$}\NormalTok{sex)) }\OperatorTok{-}\StringTok{ }\DecValTok{1}
\NormalTok{interact_l =}\StringTok{ }\KeywordTok{summary}\NormalTok{(}\KeywordTok{lm}\NormalTok{(}\DataTypeTok{data =}\NormalTok{ cps, hwage }\OperatorTok{~}\StringTok{ }\NormalTok{white }\OperatorTok{+}\StringTok{ }\NormalTok{black }\OperatorTok{+}\StringTok{ }\NormalTok{educ }\OperatorTok{+}\StringTok{ }\NormalTok{exper }\OperatorTok{+}\StringTok{ }\NormalTok{exper2 }\OperatorTok{+}\StringTok{ }\KeywordTok{I}\NormalTok{(male}\OperatorTok{*}\NormalTok{educ)))}\OperatorTok{$}\NormalTok{coefficient}
\KeywordTok{kable}\NormalTok{(interact_l)}
\end{Highlighting}
\end{Shaded}

\begin{longtable}[]{@{}lrrrr@{}}
\toprule
& Estimate & Std. Error & t value &
Pr(\textgreater{}\textbar{}t\textbar{})\tabularnewline
\midrule
\endhead
(Intercept) & 1.2723547 & 0.0315085 & 40.381267 &
0.0000000\tabularnewline
white & -0.0202005 & 0.0129536 & -1.559453 & 0.1188954\tabularnewline
black & -0.1343747 & 0.0168809 & -7.960163 & 0.0000000\tabularnewline
educ & 0.0980912 & 0.0014953 & 65.598967 & 0.0000000\tabularnewline
exper & 0.0169183 & 0.0015942 & 10.612222 & 0.0000000\tabularnewline
exper2 & -0.0002541 & 0.0000317 & -8.011397 & 0.0000000\tabularnewline
I(male * educ) & 0.0121233 & 0.0005409 & 22.414143 &
0.0000000\tabularnewline
\bottomrule
\end{longtable}

The result got from Q7 is the same the one from Q6. This suggests that
the results got from bootstap and interaction variable are the same.

\hypertarget{question-8}{%
\subsection{Question 8}\label{question-8}}

\begin{Shaded}
\begin{Highlighting}[]
\CommentTok{# Display data}
\KeywordTok{kable}\NormalTok{(}\KeywordTok{head}\NormalTok{(nlsy[,}\DecValTok{1}\OperatorTok{:}\DecValTok{8}\NormalTok{]))}
\end{Highlighting}
\end{Shaded}

\begin{longtable}[]{@{}rrrrrrrr@{}}
\toprule
age79 & foreign & urban14 & mag14 & news14 & lib14 & educ\_mom &
educ\_dad\tabularnewline
\midrule
\endhead
20 & 1 & 1 & 0 & 1 & 1 & 8 & 8\tabularnewline
20 & 1 & 1 & 1 & 1 & 1 & 5 & 8\tabularnewline
17 & 0 & 1 & 1 & 1 & 1 & 10 & 12\tabularnewline
16 & 0 & 1 & 1 & 1 & 0 & 11 & 12\tabularnewline
19 & 0 & 1 & 1 & 1 & 1 & 12 & 12\tabularnewline
18 & 0 & 1 & 0 & 1 & 1 & 12 & 12\tabularnewline
\bottomrule
\end{longtable}

\begin{Shaded}
\begin{Highlighting}[]
\CommentTok{# Unweighted summary}
\NormalTok{black =}\StringTok{ }\KeywordTok{na.omit}\NormalTok{(nlsy}\OperatorTok{$}\NormalTok{black)}
\NormalTok{hisp =}\StringTok{ }\KeywordTok{na.omit}\NormalTok{(nlsy}\OperatorTok{$}\NormalTok{hisp)}

\NormalTok{black_mean =}\StringTok{ }\KeywordTok{sum}\NormalTok{(black, }\DataTypeTok{na.rm =} \OtherTok{TRUE}\NormalTok{)}\OperatorTok{/}\KeywordTok{length}\NormalTok{(black)}
\NormalTok{hisp_mean =}\StringTok{ }\KeywordTok{sum}\NormalTok{(hisp, }\DataTypeTok{na.rm =} \OtherTok{TRUE}\NormalTok{)}\OperatorTok{/}\KeywordTok{length}\NormalTok{(hisp)}

\NormalTok{black_sd =}\StringTok{ }\NormalTok{((black_mean }\OperatorTok{*}\StringTok{ }\NormalTok{(}\DecValTok{1}\OperatorTok{-}\NormalTok{black_mean))}\OperatorTok{/}\KeywordTok{length}\NormalTok{(black))}\OperatorTok{^}\FloatTok{0.5}
\NormalTok{hisp_sd =}\StringTok{ }\NormalTok{((hisp_mean }\OperatorTok{*}\StringTok{ }\NormalTok{(}\DecValTok{1}\OperatorTok{-}\NormalTok{hisp_mean))}\OperatorTok{/}\KeywordTok{length}\NormalTok{(hisp))}\OperatorTok{^}\FloatTok{0.5}


\CommentTok{#Weighted summary}
\NormalTok{black_mean_weight =}\StringTok{ }\KeywordTok{sum}\NormalTok{(black}\OperatorTok{*}\NormalTok{nlsy}\OperatorTok{$}\NormalTok{perweight, }\DataTypeTok{na.rm =} \OtherTok{TRUE}\NormalTok{)}\OperatorTok{/}\KeywordTok{sum}\NormalTok{(nlsy}\OperatorTok{$}\NormalTok{perweight)}
\NormalTok{hisp_mean_weight =}\StringTok{ }\KeywordTok{sum}\NormalTok{(hisp}\OperatorTok{*}\NormalTok{nlsy}\OperatorTok{$}\NormalTok{perweight, }\DataTypeTok{na.rm =} \OtherTok{TRUE}\NormalTok{)}\OperatorTok{/}\KeywordTok{sum}\NormalTok{(nlsy}\OperatorTok{$}\NormalTok{perweight)}

\NormalTok{black_sd_weight =}\StringTok{ }\NormalTok{((black_mean_weight }\OperatorTok{*}\StringTok{ }\NormalTok{(}\DecValTok{1}\OperatorTok{-}\NormalTok{black_mean_weight))}\OperatorTok{/}\KeywordTok{length}\NormalTok{(black))}\OperatorTok{^}\FloatTok{0.5}
\NormalTok{hisp_sd_weight =}\StringTok{ }\NormalTok{((hisp_mean_weight }\OperatorTok{*}\StringTok{ }\NormalTok{(}\DecValTok{1}\OperatorTok{-}\NormalTok{hisp_mean_weight))}\OperatorTok{/}\KeywordTok{length}\NormalTok{(hisp))}\OperatorTok{^}\FloatTok{0.5}

\NormalTok{Q8 =}\StringTok{ }\KeywordTok{data.frame}\NormalTok{(}
  \DataTypeTok{black_mean =} \KeywordTok{sum}\NormalTok{(black, }\DataTypeTok{na.rm =} \OtherTok{TRUE}\NormalTok{)}\OperatorTok{/}\KeywordTok{length}\NormalTok{(black),}
  \DataTypeTok{hisp_mean =} \KeywordTok{sum}\NormalTok{(hisp, }\DataTypeTok{na.rm =} \OtherTok{TRUE}\NormalTok{)}\OperatorTok{/}\KeywordTok{length}\NormalTok{(hisp),}
  
  \DataTypeTok{black_sd =}\NormalTok{ ((black_mean }\OperatorTok{*}\StringTok{ }\NormalTok{(}\DecValTok{1}\OperatorTok{-}\NormalTok{black_mean))}\OperatorTok{/}\KeywordTok{length}\NormalTok{(black))}\OperatorTok{^}\FloatTok{0.5}\NormalTok{,}
  \DataTypeTok{hisp_sd =}\NormalTok{ ((hisp_mean }\OperatorTok{*}\StringTok{ }\NormalTok{(}\DecValTok{1}\OperatorTok{-}\NormalTok{hisp_mean))}\OperatorTok{/}\KeywordTok{length}\NormalTok{(hisp))}\OperatorTok{^}\FloatTok{0.5}\NormalTok{,}
  
  
  \CommentTok{#Weighted summary}
  \DataTypeTok{black_mean_weight =} \KeywordTok{sum}\NormalTok{(black}\OperatorTok{*}\NormalTok{nlsy}\OperatorTok{$}\NormalTok{perweight, }\DataTypeTok{na.rm =} \OtherTok{TRUE}\NormalTok{)}\OperatorTok{/}\KeywordTok{sum}\NormalTok{(nlsy}\OperatorTok{$}\NormalTok{perweight),}
  \DataTypeTok{hisp_mean_weight =} \KeywordTok{sum}\NormalTok{(hisp}\OperatorTok{*}\NormalTok{nlsy}\OperatorTok{$}\NormalTok{perweight, }\DataTypeTok{na.rm =} \OtherTok{TRUE}\NormalTok{)}\OperatorTok{/}\KeywordTok{sum}\NormalTok{(nlsy}\OperatorTok{$}\NormalTok{perweight),}
  
  \DataTypeTok{black_sd_weight =}\NormalTok{ ((black_mean_weight }\OperatorTok{*}\StringTok{ }\NormalTok{(}\DecValTok{1}\OperatorTok{-}\NormalTok{black_mean_weight))}\OperatorTok{/}\KeywordTok{length}\NormalTok{(black))}\OperatorTok{^}\FloatTok{0.5}\NormalTok{,}
  \DataTypeTok{hisp_sd_weight =}\NormalTok{ ((hisp_mean_weight }\OperatorTok{*}\StringTok{ }\NormalTok{(}\DecValTok{1}\OperatorTok{-}\NormalTok{hisp_mean_weight))}\OperatorTok{/}\KeywordTok{length}\NormalTok{(hisp))}\OperatorTok{^}\FloatTok{0.5}\NormalTok{)}

\KeywordTok{kable}\NormalTok{(Q8)}
\end{Highlighting}
\end{Shaded}

\begin{longtable}[]{@{}rrrrrrrr@{}}
\toprule
black\_mean & hisp\_mean & black\_sd & hisp\_sd & black\_mean\_weight &
hisp\_mean\_weight & black\_sd\_weight & hisp\_sd\_weight\tabularnewline
\midrule
\endhead
0.2501971 & 0.1578118 & 0.0038455 & 0.0032368 & 0.1418509 & 0.0654364 &
0.0030977 & 0.0021956\tabularnewline
\bottomrule
\end{longtable}

The weighted summary better discribes the population ratio becuase in
the unweighted sampings, due to the fact that black and hispanics are
Ethnic minorities, they are upsampled and therefore over-represented
Considering weight can eliminate the differences.

\hypertarget{question-9}{%
\subsection{Question 9}\label{question-9}}

\begin{Shaded}
\begin{Highlighting}[]
\NormalTok{hour_wage =}\StringTok{ }\KeywordTok{log}\NormalTok{(nlsy}\OperatorTok{$}\NormalTok{laborinc07}\OperatorTok{/}\NormalTok{nlsy}\OperatorTok{$}\NormalTok{hours07)}
\NormalTok{hour_wage[hour_wage }\OperatorTok{==}\StringTok{ }\OperatorTok{-}\OtherTok{Inf}\NormalTok{] =}\StringTok{ }\DecValTok{0}

\NormalTok{experience =}\StringTok{ }\NormalTok{nlsy}\OperatorTok{$}\NormalTok{age79 }\OperatorTok{+}\StringTok{ }\DecValTok{28} \OperatorTok{-}\StringTok{ }\NormalTok{nlsy}\OperatorTok{$}\NormalTok{educ }\OperatorTok{-}\StringTok{ }\DecValTok{5}

\NormalTok{nlsy_plus =}\StringTok{ }\KeywordTok{cbind}\NormalTok{(nlsy, experience, hour_wage)}
\NormalTok{nlsy_plus =}\StringTok{ }\KeywordTok{subset}\NormalTok{(nlsy_plus, hours07 }\OperatorTok{>}\StringTok{ }\DecValTok{1750}\NormalTok{)}
\NormalTok{nlsy_plus =}\StringTok{ }\KeywordTok{na.omit}\NormalTok{(nlsy_plus)}

\KeywordTok{rm}\NormalTok{(experience, hour_wage)}

\CommentTok{# For black}
\KeywordTok{print}\NormalTok{(}\StringTok{"Unweighted summary for the black"}\NormalTok{)}
\end{Highlighting}
\end{Shaded}

\begin{verbatim}
## [1] "Unweighted summary for the black"
\end{verbatim}

\begin{Shaded}
\begin{Highlighting}[]
\KeywordTok{kable}\NormalTok{(}\KeywordTok{summary}\NormalTok{(}\KeywordTok{lm}\NormalTok{(}\DataTypeTok{data =}\NormalTok{ nlsy_plus, hour_wage }\OperatorTok{~}\StringTok{ }\NormalTok{educ }\OperatorTok{+}\StringTok{ }
\StringTok{                   }\KeywordTok{I}\NormalTok{(experience}\OperatorTok{^}\DecValTok{2}\NormalTok{) }\OperatorTok{+}\StringTok{ }
\StringTok{                   }\NormalTok{experience }\OperatorTok{+}\StringTok{ }\NormalTok{black }\OperatorTok{+}\StringTok{ }\NormalTok{male))}\OperatorTok{$}\NormalTok{coefficient)}
\end{Highlighting}
\end{Shaded}

\begin{longtable}[]{@{}lrrrr@{}}
\toprule
& Estimate & Std. Error & t value &
Pr(\textgreater{}\textbar{}t\textbar{})\tabularnewline
\midrule
\endhead
(Intercept) & 2.7016775 & 0.6577952 & 4.107171 &
0.0000409\tabularnewline
educ & 0.1255954 & 0.0084000 & 14.951855 & 0.0000000\tabularnewline
I(experience\^{}2) & 0.0026465 & 0.0008244 & 3.210304 &
0.0013369\tabularnewline
experience & -0.1342427 & 0.0448467 & -2.993366 &
0.0027769\tabularnewline
black & -0.2188915 & 0.0317476 & -6.894745 & 0.0000000\tabularnewline
male & 0.2645347 & 0.0278221 & 9.508090 & 0.0000000\tabularnewline
\bottomrule
\end{longtable}

\begin{Shaded}
\begin{Highlighting}[]
\KeywordTok{print}\NormalTok{(}\StringTok{"Weighted summary for the black"}\NormalTok{)}
\end{Highlighting}
\end{Shaded}

\begin{verbatim}
## [1] "Weighted summary for the black"
\end{verbatim}

\begin{Shaded}
\begin{Highlighting}[]
\KeywordTok{kable}\NormalTok{(}\KeywordTok{summary}\NormalTok{(}\KeywordTok{lm}\NormalTok{(}\DataTypeTok{data =}\NormalTok{ nlsy_plus, hour_wage }\OperatorTok{~}\StringTok{ }\NormalTok{educ }\OperatorTok{+}\StringTok{ }
\StringTok{                   }\KeywordTok{I}\NormalTok{(experience}\OperatorTok{^}\DecValTok{2}\NormalTok{) }\OperatorTok{+}\StringTok{ }
\StringTok{                   }\NormalTok{experience }\OperatorTok{+}\StringTok{ }\NormalTok{black }\OperatorTok{+}\StringTok{ }\NormalTok{male, }\DataTypeTok{weights =}\NormalTok{ perweight))}\OperatorTok{$}\NormalTok{coefficient)}
\end{Highlighting}
\end{Shaded}

\begin{longtable}[]{@{}lrrrr@{}}
\toprule
& Estimate & Std. Error & t value &
Pr(\textgreater{}\textbar{}t\textbar{})\tabularnewline
\midrule
\endhead
(Intercept) & 2.3082053 & 0.6918313 & 3.336370 &
0.0008569\tabularnewline
educ & 0.1304702 & 0.0082055 & 15.900332 & 0.0000000\tabularnewline
I(experience\^{}2) & 0.0023259 & 0.0008866 & 2.623437 &
0.0087394\tabularnewline
experience & -0.1136084 & 0.0478590 & -2.373817 &
0.0176545\tabularnewline
black & -0.2308859 & 0.0447666 & -5.157550 & 0.0000003\tabularnewline
male & 0.3060472 & 0.0279673 & 10.943041 & 0.0000000\tabularnewline
\bottomrule
\end{longtable}

\begin{Shaded}
\begin{Highlighting}[]
\CommentTok{# For Hispanic}
\KeywordTok{print}\NormalTok{(}\StringTok{"Unweighted summary for the hispanic"}\NormalTok{)}
\end{Highlighting}
\end{Shaded}

\begin{verbatim}
## [1] "Unweighted summary for the hispanic"
\end{verbatim}

\begin{Shaded}
\begin{Highlighting}[]
\KeywordTok{kable}\NormalTok{(}\KeywordTok{summary}\NormalTok{(}\KeywordTok{lm}\NormalTok{(}\DataTypeTok{data =}\NormalTok{ nlsy_plus, hour_wage }\OperatorTok{~}\StringTok{ }\NormalTok{educ }\OperatorTok{+}\StringTok{ }
\StringTok{                   }\KeywordTok{I}\NormalTok{(experience}\OperatorTok{^}\DecValTok{2}\NormalTok{) }\OperatorTok{+}\StringTok{ }
\StringTok{                   }\NormalTok{experience }\OperatorTok{+}\StringTok{ }\NormalTok{hisp }\OperatorTok{+}\StringTok{ }\NormalTok{male))}\OperatorTok{$}\NormalTok{coefficient)}
\end{Highlighting}
\end{Shaded}

\begin{longtable}[]{@{}lrrrr@{}}
\toprule
& Estimate & Std. Error & t value &
Pr(\textgreater{}\textbar{}t\textbar{})\tabularnewline
\midrule
\endhead
(Intercept) & 2.8177221 & 0.6630552 & 4.2496040 &
0.0000219\tabularnewline
educ & 0.1273055 & 0.0084832 & 15.0067059 & 0.0000000\tabularnewline
I(experience\^{}2) & 0.0029466 & 0.0008290 & 3.5543455 &
0.0003835\tabularnewline
experience & -0.1501018 & 0.0451024 & -3.3280229 &
0.0008829\tabularnewline
hisp & 0.0238112 & 0.0375895 & 0.6334539 & 0.5264752\tabularnewline
male & 0.2765659 & 0.0279506 & 9.8948241 & 0.0000000\tabularnewline
\bottomrule
\end{longtable}

\begin{Shaded}
\begin{Highlighting}[]
\KeywordTok{print}\NormalTok{(}\StringTok{"Weighted summary for the hispanic"}\NormalTok{)}
\end{Highlighting}
\end{Shaded}

\begin{verbatim}
## [1] "Weighted summary for the hispanic"
\end{verbatim}

\begin{Shaded}
\begin{Highlighting}[]
\KeywordTok{kable}\NormalTok{(}\KeywordTok{summary}\NormalTok{(}\KeywordTok{lm}\NormalTok{(}\DataTypeTok{data =}\NormalTok{ nlsy_plus, hour_wage }\OperatorTok{~}\StringTok{ }\NormalTok{educ }\OperatorTok{+}\StringTok{ }
\StringTok{                   }\KeywordTok{I}\NormalTok{(experience}\OperatorTok{^}\DecValTok{2}\NormalTok{) }\OperatorTok{+}\StringTok{ }
\StringTok{                   }\NormalTok{experience }\OperatorTok{+}\StringTok{ }\NormalTok{hisp }\OperatorTok{+}\StringTok{ }\NormalTok{male, }\DataTypeTok{weights =}\NormalTok{ perweight))}\OperatorTok{$}\NormalTok{coefficient)}
\end{Highlighting}
\end{Shaded}

\begin{longtable}[]{@{}lrrrr@{}}
\toprule
& Estimate & Std. Error & t value &
Pr(\textgreater{}\textbar{}t\textbar{})\tabularnewline
\midrule
\endhead
(Intercept) & 2.3432575 & 0.6946602 & 3.3732429 &
0.0007503\tabularnewline
educ & 0.1317033 & 0.0082426 & 15.9783250 & 0.0000000\tabularnewline
I(experience\^{}2) & 0.0024453 & 0.0008896 & 2.7487750 &
0.0060100\tabularnewline
experience & -0.1198860 & 0.0480238 & -2.4963905 &
0.0125883\tabularnewline
hisp & 0.0041861 & 0.0625679 & 0.0669041 & 0.9466615\tabularnewline
male & 0.3130156 & 0.0280345 & 11.1653726 & 0.0000000\tabularnewline
\bottomrule
\end{longtable}

The usage of sampling weight can slightly change the coefficient and
increase the standard error of the statistics. However, Weighted
regression is preferred because it takes consideration of the Over
sampling of minority populations such as hispanic or black. Therefore, I
prefer the weighted regression.

\hypertarget{question-10}{%
\subsection{Question 10}\label{question-10}}

\begin{Shaded}
\begin{Highlighting}[]
\KeywordTok{ggplot}\NormalTok{(}\DataTypeTok{data =}\NormalTok{ nlsy_plus, }\KeywordTok{aes}\NormalTok{(}\DataTypeTok{x =}\NormalTok{ experience, }\DataTypeTok{y =}\NormalTok{ hour_wage)) }\OperatorTok{+}\StringTok{ }
\StringTok{  }\KeywordTok{geom_point}\NormalTok{() }\OperatorTok{+}\StringTok{ }
\StringTok{  }\KeywordTok{geom_smooth}\NormalTok{(}\DataTypeTok{method =} \StringTok{"lm"}\NormalTok{, }\DataTypeTok{color =} \StringTok{"black"}\NormalTok{) }\OperatorTok{+}\StringTok{ }
\StringTok{  }\KeywordTok{xlab}\NormalTok{(}\StringTok{"Experience"}\NormalTok{) }\OperatorTok{+}\StringTok{ }
\StringTok{  }\KeywordTok{ylab}\NormalTok{(}\StringTok{"Log(Hourly Wage)"}\NormalTok{)}
\end{Highlighting}
\end{Shaded}

\includegraphics{HW1_files/figure-latex/Q10-1.pdf}

The coefficient of education obtained from NLSY is similar to the
coefficient of education obtained from CPS. Both are similar to around
0.13. However, the coefficients of experience and experience\^{}2 are
different acrosstwo datasets. Particularly, the coefficients from NLSY
is negative, which is realistically unlikely. This suggests that either:
- The two samples are innately different in terms of populaiton
component - The way how part-time workers and non-working force are
excluded influences the result. In NLSY, workers who work for less than
1750 hours are dropped however in CPS, workers who work less than 35
hours per week or less than 50 weeks per year are dropped.

\hypertarget{question-11}{%
\subsection{Question 11}\label{question-11}}

I think \(\beta_1\) does not represent the causal effect of education.
This is because although the correlation between education and hour wage
is found to be significant. There is no evidence indicating causality.
In fact, there may be many other variables correlated with both
education and wage that are ommited in this regression For instance,
family income is traditionally believed to be correlated with education
because richer households tend to afford more education. Wealth status
is also related to wage given that difference in parents' income may
suggest difference in access to resources. Therefore, we cannot prove
that the regression is not subjective to ommited variable bias and we
cannot determin causal correlation.

\hypertarget{question-12}{%
\subsection{Question 12}\label{question-12}}

\begin{longtable}[]{@{}lrrrr@{}}
\toprule
& Estimate & Std. Error & t value &
Pr(\textgreater{}\textbar{}t\textbar{})\tabularnewline
\midrule
\endhead
(Intercept) & 2.7527504 & 0.6590189 & 4.177043 &
0.0000302\tabularnewline
educ & 0.1245644 & 0.0084399 & 14.758978 & 0.0000000\tabularnewline
I(experience\^{}2) & 0.0026805 & 0.0008248 & 3.249907 &
0.0011644\tabularnewline
experience & -0.1361035 & 0.0448682 & -3.033405 &
0.0024344\tabularnewline
black & -0.2298632 & 0.0329400 & -6.978241 & 0.0000000\tabularnewline
hisp & -0.0483787 & 0.0387629 & -1.248066 & 0.2120835\tabularnewline
male & 0.2630655 & 0.0278449 & 9.447518 & 0.0000000\tabularnewline
\bottomrule
\end{longtable}

\begin{longtable}[]{@{}lrrrr@{}}
\toprule
& Estimate & Std. Error & t value &
Pr(\textgreater{}\textbar{}t\textbar{})\tabularnewline
\midrule
\endhead
(Intercept) & 3.7232917 & 0.6551837 & 5.682821 &
0.0000000\tabularnewline
educ & 0.0662963 & 0.0098559 & 6.726588 & 0.0000000\tabularnewline
I(experience\^{}2) & 0.0030033 & 0.0008127 & 3.695461 &
0.0002226\tabularnewline
experience & -0.1672974 & 0.0442697 & -3.779045 &
0.0001598\tabularnewline
black & -0.0424899 & 0.0370793 & -1.145918 & 0.2519008\tabularnewline
hisp & 0.0709117 & 0.0403266 & 1.758437 & 0.0787533\tabularnewline
male & 0.2288108 & 0.0275976 & 8.290959 & 0.0000000\tabularnewline
urban14 & 0.0660547 & 0.0331545 & 1.992329 & 0.0464061\tabularnewline
afqt81 & 0.0073370 & 0.0006767 & 10.843085 & 0.0000000\tabularnewline
\bottomrule
\end{longtable}

I think living at urban places at the age of 14 and AFQT should be
included in the regression. As the statistics shows, the coefficients of
both Urban14 and AFQT81 are significant. As an explaination, the AFQT
score is a good indicator of one's cognitive ability and should be
correlated with earning, since we expect smart people to earn more. Also
living in urban environment at a early age can be significant to access
to resource and education quality. Therefore, I expect these two factors
to be added to the regression.

\hypertarget{question-13}{%
\subsection{Question 13}\label{question-13}}

In a natural experiment or survey setting, it's hard for OLS to indicate
causal relationship because non of the conditons are randomly assigned.
Therefore, it is hard to include/control all the variables that are
potentially correlated with the error term. The coefficients are thus
only good for passvie prediction but not causation.

\end{document}
